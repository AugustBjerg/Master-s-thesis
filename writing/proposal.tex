\documentclass[12pt,onecolumn]{article}

\usepackage[a4paper, margin=1in]{geometry}
\usepackage{mathptmx}  % Times Roman with math support
\usepackage{graphicx}
\usepackage{float}
\usepackage{amsmath}
\usepackage{url}
\usepackage{hyperref}
\usepackage{booktabs}
\usepackage{textgreek}
\usepackage{caption}
\usepackage{subcaption}
\usepackage{fancyhdr}
\usepackage{titlesec}
\usepackage{enumitem}
\usepackage{makecell}
\usepackage{placeins,needspace}
\usepackage{adjustbox}
\usepackage{parskip} % Better spacing between paragraphs
\usepackage{lipsum}
\usepackage{xcolor}
\usepackage{setspace} % Makes linespacing possible


\captionsetup[table]{font=tiny,justification=centering,position=bottom}

\doublespacing % \setstretch{1.2} for custom, you can also do single or double spacing

\definecolor{mydarkgray}{gray}{0.2}


\usepackage[style=apa, backend=biber]{biblatex}
\DeclareLanguageMapping{english}{english-apa}
\addbibresource{references/proposal.bib}  

% Customize section title appearance
\titleformat{\section}{\normalfont\Large\bfseries}{\thesection}{1em}{}
\titleformat{\subsection}{\normalfont\large\bfseries}{\thesubsection}{1em}{}

\title{\textbf{[Exploring Data-Driven Approaches to Identifying Sources of Operational Efficiency in Maritime Transport]}}
\author{
August Bjerg-Heise (114300315)
}
\date{31 December 2025}
\vspace{1em}


\begin{document}

% ----- Front Page -----
\begin{titlepage}
    \centering
    \vspace*{2cm}

    \textbf{\Large{Research Proposal}}\\[1em]
    \textbf{\Huge{Identifying Sources of Operational Efficiency in Maritime Transport}}\\[3em]

    % Placeholder for Image
    % \includegraphics[width=0.6\textwidth]{your_image.png}\\[3em]

    \begin{center}
    \textbf{Student:} August Bjerg-Heise (114300315)\\[1.2em]
    \textbf{Date of Submission:} January 1st, 2025\\[1.2em]
    % TODO: update page count before hand-in
    \textbf{Number of Pages:} 10\\[1.2em]
    \end{center}

\end{titlepage}
\clearpage

\section{Introduction}

Maritime transport of physical goods (henceforth “shipping”) is the backbone of the economy that has enabled unprecedented economic development in the last two centuries. Making shipping as efficient as possible is thus of almost universal interest to the people, governments and companies of the world. This is not only the case from a cost perspective, but perhaps even more so from an environmental perspective. Shipping constitutes 3\% of global CO2 emissions, which is more than the country of Indonesia \parencite{IMO2020FourthGHGStudy, Crippa_etal_2025_GHG_emissions_world_countries}. With 87\% of shipping emissions and 50\% of voyage costs coming from the combustion of mostly fossil fuels, operational efficiency is imperative for both economic development and the sustainability of our planet \parencite{ScheerEtAl2021_TPIShippingMethodology}. 

In shipping, operational efficiency is largely a question of minimizing the amount of fuel spent to complete a voyage within a certain amount of time \parencite{Coraddu2019}. A myriad of factors such as wind conditions, engine state, cargo load and marine life influence this efficiency, making it an extremely difficult to determine the optimal operating mode of a ship at any given point in time. This often leads to methodological corner-cutting, causing most ships to operate sub-optimally. It is estimated that ships, on average, burn 15\% more fuel than what is necessary per voyage \parencite{Duan2024}. Therefore, this proposal aims to present a framework for answering the following research questions:

\textit{What are the causes and implications of operational inefficiency in maritime transport, and what can be done to mitigate it?}

\begin{enumerate}
  \item \textit{How can operational inefficiencies in shipping be quantified?}
  \item \textit{How significant are the operational inefficiencies on a vessel in operation?}
  \item \textit{What are the main causes of operational inefficiencies?}
  \item \textit{What changes can shipping companies make to optimize the operational efficiency of vessels?}
\end{enumerate}

The aim is thus \textit{not} to evaluate the viability or impact of new technologies, such as green fuels, although these are recognised as necessary components of the decarbonization of the shipping sector.

\section{Literature Review}

Ever since the first ships were built, operators have sought to optimize energy use, first in manpower and wind, and later in modern fuels. The earliest academic contributions are commonly traced to 19th-century work on the physics of ship resistance and propulsion \parencite{Russell1839Experimental}, alongside the pioneering of systematic towing-test approaches that enabled empirical inference about performance \parencite{froude1955papers}. This early literature cemented a central operational hypothesis: efficiency is governed primarily by the relationship between speed and required propulsive power, a principle embedded today in the “speed–power curve,” which maps the power needed to maintain each steady speed under specified conditions. Through almost two centuries of research, it has been found that the relationship between speed and propulsion power is approximately cubic, i.e., power need rises steeply with speed (see example in Figure \ref{fig:example_spd_pwr_curve}). 

\begin{figure}[htbp]
   \centering
   \captionsetup{font=scriptsize,justification=centering}
   \includegraphics[width=0.65\linewidth]{figures/proposal/example_spd_pwr_curve.png}
   \caption{Example of a speed-power curve derived from empirical tests \parencite{Schultz2007EffectsOfCoating}}
   \label{fig:example_spd_pwr_curve}
\end{figure}

As measurement and computation matured, subsequent research increasingly combined physics-grounded models with real operational data, enabling more mathematically complex and empirically testable approaches to operational efficiency analysis \parencite{kim2025comprehensive}. More contemporary work can be grouped into three broad methodological categories: i) Computational Fluid Dynamics, ii) Empirical Naval Architecture and iii) Data Science.

Computational fluid dynamics (CFD) is a physics-based approach that numerically solves the flow equations around a ship hull to predict resistance and related propulsion requirements as a function of speed and operating conditions \parencite{Russell1839Experimental}. Conceptually, this computational programme extends early resistance-focused science, including Russell’s experimental work that treated resistance as a mechanistic phenomenon. Contemporary studies broaden the optimization target from calm-water performance to operational conditions, for example panel-method modelling of wave added resistance and related wave-load computations that refine how power penalties are represented under realistic sea states \parencite{Dong_2023,Wang_2022}. Relative to purely empirical curves and data-driven models, CFD’s key strengths are physical interpretability and theoretical rigor, while its main limitations remain sensitivity to modelling assumptions and the need for careful verification and validation, alongside high computational cost.

Empirical naval architecture (ENA) models estimate the speed–power relationship from physical measurements, most commonly towing-tank tests where a scale model is towed at controlled speeds and resistance is measured for inference. In contrast to CFD, which derives the same relationship by numerically solving flow equations, towing tests provide benchmark data that is often used to validate and calibrate computational predictions. This experimental tradition is commonly linked to William Froude and Robert Froude’s \parencite*{froude1955papers} development of systematic model testing and scaling concepts for predicting full-scale ship performance. More recent applications include towing-tank studies of operational interventions such as hull coatings \parencite{Schultz2007EffectsOfCoating}, while the core trade-off remains that experiments can be highly accurate under controlled conditions but are expensive, time-consuming, and not always fully representative of real sea states and in-service hull condition.

Data-science approaches model operational efficiency directly from observed data, typically without imposing any physical constraints. However, physics-informed feature choice and engineering remain necessary to avoid spurious inference. Recent work leverages increasingly dense IoT (sensor) data streams to estimate performance baselines and degradation signals, for example by building data-driven digital twins that predict “expected” speed and then interpret systematic deviations as fouling-induced speed loss \parencite{Coraddu2019}. Related studies fuse operational logs with metocean data and apply machine-learning regressors to quantify hull condition, enabling counterfactual evaluation of cleaning timing under controlled operating scenarios \parencite{Duan2024}. The main strength of these methods is flexibility in capturing ship-specific, nonlinear relationships at scale as data availability and ML toolchains have improved, but their central limitation is dependence of very high volume, frequency and quality of data \parencite{kim2025comprehensive}. From a research perspective, exploring new avenues within data-science based approaches is a particularly interesting due to the recent rapid developments in the quantity and quality of data.

\section{Theoretical Framework and Hypotheses}
Given the cubic relationship between speed and propulsion power, fuel costs are usually the main barrier to the speed at which a ship completes a voyage. This was especially clear during the 1970s oil crisis and the 2008 financial crisis, where the practice of “slow steaming”, i.e., operating below design speed to cut costs, was widely adopted \parencite{Meiliana_Syahab_Wulandari_Utama_2024}. This means that quantifying operational efficiency is not trivial, as it is not only a question of how much fuel is used, but also how fast the ship completes the voyage (and thus the shipping costs that the shipping company can charge). However, it is widely agreed that anything that can be said to shift the speed-power curve left (i.e., less power required for a given speed) is an improvement in operational efficiency \parencite{Coraddu2019}. This means that quantifying inefficiency at a given point in time is complex, since the speed-power curve gradually changes based on the ship’s condition and state \parencite{Schultz2007EffectsOfCoating}. Scholars often circumvent this issue by comparing performance to a baseline model representing the theoretically optimal speed of the vessel \parencite{Coraddu2019} at a given time.

Contemporary baseline models for vessel performance are, with few exceptions, derived from physics-based representations of ship resistance and propulsion, where the baseline speed–power relation is constructed from an assumed functional form and calibrated coefficients \parencite{ ISO19030-2-2016}. In practice, these formulations typically decompose the required propulsive power into a small set of components representing the dominant resistance contributions and propulsion efficiency terms, and they are then evaluated under reference conditions to define an “expected” speed or power level. By contrast, a purely data-driven baseline relies on two prerequisites. First, the dataset must contain sufficient information to represent the main processes that drive speed loss, which in turn requires broad sensor coverage and consistent recording of operational and environmental variables. Second, the data must include a period that can credibly be treated as near-optimal operation, so that the estimated baseline reflects the vessel’s dynamics in the absence of the inefficiencies under study rather than embedding them into the reference \parencite{Coraddu2019}.

Based on reviewed literature and the theoretical considerations above, the following hypotheses are proposed:

\begin{itemize}
   \item \textbf{H1}: A ship in operation sails about 15\% slower than what is theoretically possible at a given fuel consumption
   \item \textbf{H2}: The main sources of inefficiencies are biofouling, weather conditions and engine depreciation
   \item \textbf{H3}: Statistical methods will at least match traditional methods in their ability to predict and attribute operational inefficiencies in real-time
\end{itemize}

\section{Dataset}
\subsection{Origin}

A case-study approach is adopted for two reasons. First, the shipping industry lacks a globally agreed standard for data formats and collection practices, which makes cross-vessel analysis difficult and increases the risk of unobservable biases in comparative datasets \parencite{ISO19030-2-2016}. Second, many vessels are not equipped with the sensor infrastructure required for high-frequency operational performance modelling, making the construction of a consistent, high-quality multi-vessel dataset unrealistic in practice \parencite{kim2025comprehensive}.

The dataset used in this study consists of one year of time-series measurements from a bulk carrier in operation. It was provided through the “Real Vessel Data Challenge”, a workshop hosted by the Mærsk McKinney-Møller Center for Zero Carbon Shipping.\footnote{Since this is the research i will conduct for my master’s thesis, access to the dataset and workshop specifications has been granted, but these materials are not publicly accessible.} \parencite{TsarsitalidisPonkratovTsoulakos2025RealVesselDataChallenge}. The data cover both voyages and port operations in roughly equal shares, while the vessel’s identity and routes remain undisclosed.

The dataset combines ship particulars with operational time-series data and external conditions. Key ship particulars include length overall (224.92 m), breadth (32.25 m), typical operating drafts (about 12.2 m at design and 14.25 m at scantling), main engine maximum continuous rating (SMCR) of 9,500 kW at 105 rpm, and a propeller diameter of 6.8 m. The operational dataset contains 33 variables, with ship-internal measurements recorded at 15-second intervals, including speed over ground, speed through water, shaft power, shaft rpm, shaft torque, fuel oil mass flow, and draft. External conditions are provided at 1-hour intervals and include wind speed and direction, significant wave height, swell significant height, wave period, sea-water temperature, and current components \parencite{TsarsitalidisPonkratovTsoulakos2025RealVesselDataChallenge}.

\section{Methodology}

\label{sec:modelling_framework}
\subsection{Modelling framework}

It is intended to use statistical modelling to separate an estimated “optimal” operating speed from observed performance. First, an “optimal-speed” benchmark will be constructed by fitting a predictive model of vessel speed using data from a period in which operation can reasonably be assumed close to optimal, such as the first month following a hull and propeller cleaning event. This benchmark model will then be used to compute speed loss at each operating point as the difference between predicted optimal speed and observed speed. Second, a separate explanatory model will be estimated on data from a later, non-optimal period, in which fouling, engine depreciation, and other degradation mechanisms are expected to be present, with the dependent variable defined as the estimated speed loss and the regressors given by operational and environmental covariates in the dataset. The latter will be built for an inferential purpose, i.e., the primary purpose will be to deconstruct to infer what variables led to a given prediction. This will allow for the attribution of developments in inefficiency to certain variables and, in extension, parts of the ship. The approach to modelling is shown in \ref{fig:outdated_approach_viz}.
    
\begin{figure}[htbp]
   \centering
   \captionsetup{font=scriptsize,justification=centering}
   \includegraphics[width=0.85\linewidth]{figures/proposal/outdated_approach_viz.png}
   \caption{Visualisation of approach}
   \label{fig:outdated_approach_viz}
\end{figure}

\subsection{models}

To predict the speed loss, It is suggested to employ and compare multiple models of varying complexity for two reasons. Firstly, it increases the likelihood of achieving a satisfactory predictive power. Secondly, it allows for the evaluation of the accuracy/complexity trade-off which is particularly relevant in edge computing scenarios such as onboard deployment. Applying three different models are suggested as a starting point. As a simple baseline model, it is suggested to use a linear regression due to its simplicity and intuitive interpretation of coefficients. As a natural progression of complexity, it is suggested to use a decision tree since it can model more complex relationships while still staying relatively interpretable. Lastly, it is suggested to include neural network-based model to have a high complexity benchmark to inform the complexity/trade-off analysis. The best predictive model should be chosen for a inferential analysis as described in Section \ref{sec:modelling_framework}. It is suggested to take an experimental approach and also look into other models to find the best model.

For interpretation and attribution, the inferential analysis will be tied to the best-performing predictive model, while maintaining a consistent attribution framework across model classes. For the linear regression, attribution follows directly from estimated coefficients, interpreted as marginal associations conditional on the included covariates and modelling assumptions. For the decision tree model, local and global drivers of predicted speed loss will be quantified using TreeSHAP, which decomposes each prediction into additive feature contributions while preserving consistency properties for tree-based models. For the neural-network model, SHAP values will be used analogously to obtain feature-level contributions to each predicted speed-loss outcome, enabling a comparable attribution of operational and environmental drivers even under highly nonlinear specifications.

\section{Expected Results and Applications (1.5 page)}

\clearpage

\printbibliography

\clearpage
\appendix

\section*{Appendix}
\addcontentsline{toc}{section}{Appendix}

% Make appendix sections use Arabic numbers instead of letters:
\renewcommand{\thesection}{\arabic{section}}
\setcounter{section}{0}

\section{[Appendix Section Title]}
\label{sec:appendix1}

[Your appendix content goes here.]

% Example of appendix figure
% \begin{figure}[htbp]
%   \centering
%   \includegraphics[width=0.8\linewidth]{appendix_figure.png}
%   \caption{Your appendix figure caption}
%   \label{fig:appendix_label}
% \end{figure}

\end{document}
